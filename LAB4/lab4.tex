\documentclass[11pt]{article}
 
\usepackage[T1]{fontenc}
\usepackage[polish]{babel}
\usepackage[utf8]{inputenc}
\usepackage{lmodern}
\usepackage{graphicx}
\selectlanguage{polish}

\begin{document}
\begin{titlepage}
\title{Sortowanie - Laboratorium nr 4 z PAMSI}
\author{Justyna Klijewska}
\date{23 03 2014}
\maketitle
\end{titlepage}

\underline{Zadanie do wykonania}

W istniejącym już programie dodać sortowanie.
\\ \\ \\
Program był pisany w środowisku Windows. Jest to wersja na ocenę 5. Zostały zrealizowane sortowania merge sort, quicksort i heap sort. Porównanie ich złożoności obliczeniowej znajduje się poniżej. \\ \\
\begin{tabular}{|r|c|l|}
  \hline 
  Sortowanie & Złożoność czasowa & Złożoność pamięciowa \\
  \hline
  Quicksort & O(nlogn) & Zależnie od implementacji \\
  \hline
  Mergesort & O(nlogn) & O(n) \\
  \hline
  Heapsort & O(nlogn) &	O(n) \\
  \hline
\end{tabular} 
\\ \\ Tabela 1. Zależności między liczbą elementów w pliku a czasem wykonywania programu.
\\
\\ \\ \\ \\ 
\textit{QUICKSORT}
\\ \\
Jest to sortowanie szybkie i polega na zasadzie "dziel i zwyciężaj". Jego złożoność obliczeniowa to O(nlogn). Jest jednym z najczęściej używanych ze względu na szybkość wykonywania oraz łatwość implementacji. \\

\begin{figure}[ht!]
\centering
\includegraphics[width=90mm]{quick.jpg}
\caption{Wykres zależności liczby elementów w pliku od czasu wykonywania sortowania}
\label{overflow}
\end{figure}


\textit{MERGESORT }
\\ \\ \\
Jest to sortowanie przez scalanie i jego złożoność obliczeniowa to O(n log n). Podobnie jak quicksort korzysta z zasady "dziel i zwyciężaj".
\\
\begin{figure}[ht!]
\centering
\includegraphics[width=90mm]{merge.jpg}
\caption{Wykres zależności liczby elementów w pliku od czasu wykonywania sortowania}
\label{overflow}
\end{figure}


\textit{HEAPSORT}
\\ \\ \\
Jest to to sortowanie przez kopcowanie. Złożoność czasowa wynosi  O(n log n), a pamięciowa – O(n). Jedną z największych jej zalet jest możliwość użycia tej samej tablicy w której znajdowały się nieposortowane elementy.
\\ \\
\begin{figure}[ht!]
\centering
\includegraphics[width=90mm]{heap.jpg}
\caption{Wykres zależności liczby elementów w pliku od czasu wykonywania sortowania}
\label{overflow}
\end{figure}
\\
\\ \\ \\ \\ \\ \\ \\ \\

\underline {WNIOSKI:} \\
Najpopularniejszym i najczęściej używanym sortowaniem jest quicksort. I ciężko się dziwić, gdzyż w powyższych testach wykonywał on najszybciej swoje zadanie. Nie zmienia to faktu, że podane sortowania mają podobną złożoność obliczeniową, a co za tym idzie podobny czas wykonywania programu. Gdyby porównać powyższe z sortowaniem np.: bąbelkowym róznice byłyby większe.
\end{document}