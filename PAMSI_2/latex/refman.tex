\documentclass{book}
\usepackage[a4paper,top=2.5cm,bottom=2.5cm,left=2.5cm,right=2.5cm]{geometry}
\usepackage{makeidx}
\usepackage{natbib}
\usepackage{graphicx}
\usepackage{multicol}
\usepackage{float}
\usepackage{listings}
\usepackage{color}
\usepackage{ifthen}
\usepackage[table]{xcolor}
\usepackage{textcomp}
\usepackage{alltt}
\usepackage{ifpdf}
\ifpdf
\usepackage[pdftex,
            pagebackref=true,
            colorlinks=true,
            linkcolor=blue,
            unicode
           ]{hyperref}
\else
\usepackage[ps2pdf,
            pagebackref=true,
            colorlinks=true,
            linkcolor=blue,
            unicode
           ]{hyperref}
\usepackage{pspicture}
\fi
\usepackage[utf8]{inputenc}
\usepackage{polski}
\usepackage[T1]{fontenc}

\usepackage{mathptmx}
\usepackage[scaled=.90]{helvet}
\usepackage{courier}
\usepackage{sectsty}
\usepackage{amssymb}
\usepackage[titles]{tocloft}
\usepackage{doxygen}
\lstset{language=C++,inputencoding=utf8,basicstyle=\footnotesize,breaklines=true,breakatwhitespace=true,tabsize=8,numbers=left }
\makeindex
\setcounter{tocdepth}{3}
\renewcommand{\footrulewidth}{0.4pt}
\renewcommand{\familydefault}{\sfdefault}
\hfuzz=15pt
\setlength{\emergencystretch}{15pt}
\hbadness=750
\tolerance=750
\begin{document}
\hypersetup{pageanchor=false,citecolor=blue}
\begin{titlepage}
\vspace*{7cm}
\begin{center}
{\Large Lab 2 P\-A\-M\-S\-I \\[1ex]\large 0.\-1 }\\
\vspace*{1cm}
{\large Wygenerowano przez Doxygen 1.8.1.2}\\
\vspace*{0.5cm}
{\small Pn, 10 mar 2014 00:11:31}\\
\end{center}
\end{titlepage}
\clearemptydoublepage
\pagenumbering{roman}
\tableofcontents
\clearemptydoublepage
\pagenumbering{arabic}
\hypersetup{pageanchor=true,citecolor=blue}
\chapter{Dokumentacja zadania P\-A\-M\-S\-I L\-A\-B 2}
\label{index}\hypertarget{index}{}\begin{DoxyAuthor}{Autor}
Justyna Klijewska 
\end{DoxyAuthor}
\begin{DoxyDate}{Data}
23.\-03.\-2014 
\end{DoxyDate}
\begin{DoxyVersion}{Wersja}
0.\-1 
\end{DoxyVersion}

\chapter{Indeks klas}
\section{Lista klas}
Tutaj znajdują się klasy, struktury, unie i interfejsy wraz z ich krótkimi opisami\-:\begin{DoxyCompactList}
\item\contentsline{section}{\hyperlink{class_tab___aso}{Tab\-\_\-\-Aso$<$ T\-\_\-\-K, T\-\_\-\-D $>$} \\*Szablon klasy obs�uguj�cy tablice asocjacyjn� opart� na li�cie jednokierunkowej gdzie elementem listy jest struktura \hyperlink{struct_tab_aso_ele}{Tab\-Aso\-Ele} }{\pageref{class_tab___aso}}{}
\item\contentsline{section}{\hyperlink{struct_tab_aso_ele}{Tab\-Aso\-Ele$<$ T\-\_\-\-K, T\-\_\-\-D $>$} \\*Szablon struktury element�w tablicy asocjacyjnej }{\pageref{struct_tab_aso_ele}}{}
\end{DoxyCompactList}

\chapter{Indeks plików}
\section{File List}
Here is a list of all files with brief descriptions\-:\begin{DoxyCompactList}
\item\contentsline{section}{C\-:/\-Users/\-Klijek/\-Documents/\-Git\-Hub/\-Pamis02/\-L\-A\-B8/prj/\hyperlink{graf_8cpp}{graf.\-cpp} }{\pageref{graf_8cpp}}{}
\item\contentsline{section}{C\-:/\-Users/\-Klijek/\-Documents/\-Git\-Hub/\-Pamis02/\-L\-A\-B8/prj/\hyperlink{graf_8hpp}{graf.\-hpp} }{\pageref{graf_8hpp}}{}
\item\contentsline{section}{C\-:/\-Users/\-Klijek/\-Documents/\-Git\-Hub/\-Pamis02/\-L\-A\-B8/prj/\hyperlink{main_8cpp}{main.\-cpp} }{\pageref{main_8cpp}}{}
\end{DoxyCompactList}

\chapter{Dokumentacja klas}
\hypertarget{class_tablica}{\section{Dokumentacja klasy Tablica}
\label{class_tablica}\index{Tablica@{Tablica}}
}


{\ttfamily \#include $<$tablica.\-hpp$>$}

\subsection*{Metody publiczne}
\begin{DoxyCompactItemize}
\item 
void \hyperlink{class_tablica_a19ca2c588d95ca40a83a3078318bc1ec}{Czytaj} (char $\ast$plik)
\item 
void \hyperlink{class_tablica_a6743db90faaa2a5df74368a90e5a3db4}{Dodaj\-Na\-Koniec} (double ele)
\item 
void \hyperlink{class_tablica_a2f5ae8c1d478b1cdb62faf024940ae07}{Dodaj\-Na\-Poczatek} (double ele)
\item 
double \hyperlink{class_tablica_abc7e36bd34b5bc4a8accfd28515ab096}{Pobierz\-Pierwszy\-Ele} ()
\item 
double \hyperlink{class_tablica_a025f70f455157d7e00328ee3a7740208}{Pobierz\-Ostatni\-Ele} ()
\item 
double \hyperlink{class_tablica_aebdf0ddec6760f83973cee069952e0f3}{Pobierz\-Rozmiar} ()
\item 
void \hyperlink{class_tablica_af1d87fe0f3e512738e92a3d93e96915f}{Zamien\-Ele} (int pierwszy, int drugi)
\item 
bool \hyperlink{class_tablica_a7b1b0165c4412dca32593a085f6d079b}{Odwroc\-Tab} ()
\item 
bool \hyperlink{class_tablica_ad64a404d8b30bda2e31a8948cf71974e}{Dodaj\-Na\-Miejsce} (double ele, int miejsce)
\item 
double \hyperlink{class_tablica_a3889565193c1c9724b6055c0e5fae3c5}{Pobierz\-Wskazany\-Ele} (int miejsce)
\item 
void \hyperlink{class_tablica_a64ede62c24b4ac458df98a0705f0a96f}{Wyswietl\-Tab} ()
\item 
bool \hyperlink{class_tablica_a89fe0ecccfbc932c994bb415f9c78931}{Usun\-Ele} (int miejsce)
\item 
\hyperlink{class_tablica}{Tablica} \hyperlink{class_tablica_a15c072e7160bfbdbc5d103cf0ebd6e76}{operator+} (\hyperlink{class_tablica}{Tablica} \&T2)
\item 
\hyperlink{class_tablica}{Tablica} \hyperlink{class_tablica_a53bd7c9853f01a78ba2aff61ece4ccbf}{operator=} (\hyperlink{class_tablica}{Tablica} \&T2)
\item 
\hyperlink{class_tablica}{Tablica} \hyperlink{class_tablica_ae5d9fdf31df882eae683abc89fec01ad}{operator==} (\hyperlink{class_tablica}{Tablica} \&T2)
\end{DoxyCompactItemize}


\subsection{Opis szczegółowy}


Definicja w linii 14 pliku tablica.\-hpp.



\subsection{Dokumentacja funkcji składowych}
\hypertarget{class_tablica_a19ca2c588d95ca40a83a3078318bc1ec}{\index{Tablica@{Tablica}!Czytaj@{Czytaj}}
\index{Czytaj@{Czytaj}!Tablica@{Tablica}}
\subsubsection[{Czytaj}]{\setlength{\rightskip}{0pt plus 5cm}void Tablica\-::\-Czytaj (
\begin{DoxyParamCaption}
\item[{char $\ast$}]{plik}
\end{DoxyParamCaption}
)}}\label{class_tablica_a19ca2c588d95ca40a83a3078318bc1ec}


Definicja w linii 241 pliku tablica.\-cpp.



Oto graf wywoływań tej funkcji\-:
\nopagebreak
\begin{figure}[H]
\begin{center}
\leavevmode
\includegraphics[width=238pt]{class_tablica_a19ca2c588d95ca40a83a3078318bc1ec_icgraph}
\end{center}
\end{figure}


\hypertarget{class_tablica_a6743db90faaa2a5df74368a90e5a3db4}{\index{Tablica@{Tablica}!Dodaj\-Na\-Koniec@{Dodaj\-Na\-Koniec}}
\index{Dodaj\-Na\-Koniec@{Dodaj\-Na\-Koniec}!Tablica@{Tablica}}
\subsubsection[{Dodaj\-Na\-Koniec}]{\setlength{\rightskip}{0pt plus 5cm}void Tablica\-::\-Dodaj\-Na\-Koniec (
\begin{DoxyParamCaption}
\item[{double}]{ele}
\end{DoxyParamCaption}
)}}\label{class_tablica_a6743db90faaa2a5df74368a90e5a3db4}


Definicja w linii 11 pliku tablica.\-cpp.



Oto graf wywoływań tej funkcji\-:
\nopagebreak
\begin{figure}[H]
\begin{center}
\leavevmode
\includegraphics[width=350pt]{class_tablica_a6743db90faaa2a5df74368a90e5a3db4_icgraph}
\end{center}
\end{figure}


\hypertarget{class_tablica_ad64a404d8b30bda2e31a8948cf71974e}{\index{Tablica@{Tablica}!Dodaj\-Na\-Miejsce@{Dodaj\-Na\-Miejsce}}
\index{Dodaj\-Na\-Miejsce@{Dodaj\-Na\-Miejsce}!Tablica@{Tablica}}
\subsubsection[{Dodaj\-Na\-Miejsce}]{\setlength{\rightskip}{0pt plus 5cm}bool Tablica\-::\-Dodaj\-Na\-Miejsce (
\begin{DoxyParamCaption}
\item[{double}]{ele, }
\item[{int}]{miejsce}
\end{DoxyParamCaption}
)}}\label{class_tablica_ad64a404d8b30bda2e31a8948cf71974e}


Definicja w linii 122 pliku tablica.\-cpp.

\hypertarget{class_tablica_a2f5ae8c1d478b1cdb62faf024940ae07}{\index{Tablica@{Tablica}!Dodaj\-Na\-Poczatek@{Dodaj\-Na\-Poczatek}}
\index{Dodaj\-Na\-Poczatek@{Dodaj\-Na\-Poczatek}!Tablica@{Tablica}}
\subsubsection[{Dodaj\-Na\-Poczatek}]{\setlength{\rightskip}{0pt plus 5cm}void Tablica\-::\-Dodaj\-Na\-Poczatek (
\begin{DoxyParamCaption}
\item[{double}]{ele}
\end{DoxyParamCaption}
)}}\label{class_tablica_a2f5ae8c1d478b1cdb62faf024940ae07}


Definicja w linii 22 pliku tablica.\-cpp.

\hypertarget{class_tablica_a7b1b0165c4412dca32593a085f6d079b}{\index{Tablica@{Tablica}!Odwroc\-Tab@{Odwroc\-Tab}}
\index{Odwroc\-Tab@{Odwroc\-Tab}!Tablica@{Tablica}}
\subsubsection[{Odwroc\-Tab}]{\setlength{\rightskip}{0pt plus 5cm}bool Tablica\-::\-Odwroc\-Tab (
\begin{DoxyParamCaption}
{}
\end{DoxyParamCaption}
)}}\label{class_tablica_a7b1b0165c4412dca32593a085f6d079b}


Definicja w linii 101 pliku tablica.\-cpp.



Oto graf wywoływań tej funkcji\-:
\nopagebreak
\begin{figure}[H]
\begin{center}
\leavevmode
\includegraphics[width=350pt]{class_tablica_a7b1b0165c4412dca32593a085f6d079b_icgraph}
\end{center}
\end{figure}


\hypertarget{class_tablica_a15c072e7160bfbdbc5d103cf0ebd6e76}{\index{Tablica@{Tablica}!operator+@{operator+}}
\index{operator+@{operator+}!Tablica@{Tablica}}
\subsubsection[{operator+}]{\setlength{\rightskip}{0pt plus 5cm}{\bf Tablica} Tablica\-::operator+ (
\begin{DoxyParamCaption}
\item[{{\bf Tablica} \&}]{T2}
\end{DoxyParamCaption}
)}}\label{class_tablica_a15c072e7160bfbdbc5d103cf0ebd6e76}


Definicja w linii 178 pliku tablica.\-cpp.

\hypertarget{class_tablica_a53bd7c9853f01a78ba2aff61ece4ccbf}{\index{Tablica@{Tablica}!operator=@{operator=}}
\index{operator=@{operator=}!Tablica@{Tablica}}
\subsubsection[{operator=}]{\setlength{\rightskip}{0pt plus 5cm}{\bf Tablica} Tablica\-::operator= (
\begin{DoxyParamCaption}
\item[{{\bf Tablica} \&}]{T2}
\end{DoxyParamCaption}
)}}\label{class_tablica_a53bd7c9853f01a78ba2aff61ece4ccbf}


Definicja w linii 193 pliku tablica.\-cpp.

\hypertarget{class_tablica_ae5d9fdf31df882eae683abc89fec01ad}{\index{Tablica@{Tablica}!operator==@{operator==}}
\index{operator==@{operator==}!Tablica@{Tablica}}
\subsubsection[{operator==}]{\setlength{\rightskip}{0pt plus 5cm}{\bf Tablica} Tablica\-::operator== (
\begin{DoxyParamCaption}
\item[{{\bf Tablica} \&}]{T2}
\end{DoxyParamCaption}
)}}\label{class_tablica_ae5d9fdf31df882eae683abc89fec01ad}


Definicja w linii 204 pliku tablica.\-cpp.

\hypertarget{class_tablica_a025f70f455157d7e00328ee3a7740208}{\index{Tablica@{Tablica}!Pobierz\-Ostatni\-Ele@{Pobierz\-Ostatni\-Ele}}
\index{Pobierz\-Ostatni\-Ele@{Pobierz\-Ostatni\-Ele}!Tablica@{Tablica}}
\subsubsection[{Pobierz\-Ostatni\-Ele}]{\setlength{\rightskip}{0pt plus 5cm}double Tablica\-::\-Pobierz\-Ostatni\-Ele (
\begin{DoxyParamCaption}
{}
\end{DoxyParamCaption}
)}}\label{class_tablica_a025f70f455157d7e00328ee3a7740208}


Definicja w linii 44 pliku tablica.\-cpp.

\hypertarget{class_tablica_abc7e36bd34b5bc4a8accfd28515ab096}{\index{Tablica@{Tablica}!Pobierz\-Pierwszy\-Ele@{Pobierz\-Pierwszy\-Ele}}
\index{Pobierz\-Pierwszy\-Ele@{Pobierz\-Pierwszy\-Ele}!Tablica@{Tablica}}
\subsubsection[{Pobierz\-Pierwszy\-Ele}]{\setlength{\rightskip}{0pt plus 5cm}double Tablica\-::\-Pobierz\-Pierwszy\-Ele (
\begin{DoxyParamCaption}
{}
\end{DoxyParamCaption}
)}}\label{class_tablica_abc7e36bd34b5bc4a8accfd28515ab096}


Definicja w linii 33 pliku tablica.\-cpp.

\hypertarget{class_tablica_aebdf0ddec6760f83973cee069952e0f3}{\index{Tablica@{Tablica}!Pobierz\-Rozmiar@{Pobierz\-Rozmiar}}
\index{Pobierz\-Rozmiar@{Pobierz\-Rozmiar}!Tablica@{Tablica}}
\subsubsection[{Pobierz\-Rozmiar}]{\setlength{\rightskip}{0pt plus 5cm}double Tablica\-::\-Pobierz\-Rozmiar (
\begin{DoxyParamCaption}
{}
\end{DoxyParamCaption}
)}}\label{class_tablica_aebdf0ddec6760f83973cee069952e0f3}


Definicja w linii 55 pliku tablica.\-cpp.

\hypertarget{class_tablica_a3889565193c1c9724b6055c0e5fae3c5}{\index{Tablica@{Tablica}!Pobierz\-Wskazany\-Ele@{Pobierz\-Wskazany\-Ele}}
\index{Pobierz\-Wskazany\-Ele@{Pobierz\-Wskazany\-Ele}!Tablica@{Tablica}}
\subsubsection[{Pobierz\-Wskazany\-Ele}]{\setlength{\rightskip}{0pt plus 5cm}double Tablica\-::\-Pobierz\-Wskazany\-Ele (
\begin{DoxyParamCaption}
\item[{int}]{miejsce}
\end{DoxyParamCaption}
)}}\label{class_tablica_a3889565193c1c9724b6055c0e5fae3c5}


Definicja w linii 135 pliku tablica.\-cpp.

\hypertarget{class_tablica_a89fe0ecccfbc932c994bb415f9c78931}{\index{Tablica@{Tablica}!Usun\-Ele@{Usun\-Ele}}
\index{Usun\-Ele@{Usun\-Ele}!Tablica@{Tablica}}
\subsubsection[{Usun\-Ele}]{\setlength{\rightskip}{0pt plus 5cm}bool Tablica\-::\-Usun\-Ele (
\begin{DoxyParamCaption}
\item[{int}]{miejsce}
\end{DoxyParamCaption}
)}}\label{class_tablica_a89fe0ecccfbc932c994bb415f9c78931}


Definicja w linii 165 pliku tablica.\-cpp.

\hypertarget{class_tablica_a64ede62c24b4ac458df98a0705f0a96f}{\index{Tablica@{Tablica}!Wyswietl\-Tab@{Wyswietl\-Tab}}
\index{Wyswietl\-Tab@{Wyswietl\-Tab}!Tablica@{Tablica}}
\subsubsection[{Wyswietl\-Tab}]{\setlength{\rightskip}{0pt plus 5cm}void Tablica\-::\-Wyswietl\-Tab (
\begin{DoxyParamCaption}
{}
\end{DoxyParamCaption}
)}}\label{class_tablica_a64ede62c24b4ac458df98a0705f0a96f}


Definicja w linii 147 pliku tablica.\-cpp.



Oto graf wywoływań tej funkcji\-:
\nopagebreak
\begin{figure}[H]
\begin{center}
\leavevmode
\includegraphics[width=266pt]{class_tablica_a64ede62c24b4ac458df98a0705f0a96f_icgraph}
\end{center}
\end{figure}


\hypertarget{class_tablica_af1d87fe0f3e512738e92a3d93e96915f}{\index{Tablica@{Tablica}!Zamien\-Ele@{Zamien\-Ele}}
\index{Zamien\-Ele@{Zamien\-Ele}!Tablica@{Tablica}}
\subsubsection[{Zamien\-Ele}]{\setlength{\rightskip}{0pt plus 5cm}void Tablica\-::\-Zamien\-Ele (
\begin{DoxyParamCaption}
\item[{int}]{pierwszy, }
\item[{int}]{drugi}
\end{DoxyParamCaption}
)}}\label{class_tablica_af1d87fe0f3e512738e92a3d93e96915f}


Definicja w linii 68 pliku tablica.\-cpp.



Oto graf wywoływań tej funkcji\-:
\nopagebreak
\begin{figure}[H]
\begin{center}
\leavevmode
\includegraphics[width=350pt]{class_tablica_af1d87fe0f3e512738e92a3d93e96915f_icgraph}
\end{center}
\end{figure}




Dokumentacja dla tej klasy została wygenerowana z plików\-:\begin{DoxyCompactItemize}
\item 
prj/\hyperlink{tablica_8hpp}{tablica.\-hpp}\item 
prj/\hyperlink{tablica_8cpp}{tablica.\-cpp}\end{DoxyCompactItemize}

\hypertarget{class_zegar}{\section{Zegar Class Reference}
\label{class_zegar}\index{Zegar@{Zegar}}
}
\subsection*{Public Member Functions}
\begin{DoxyCompactItemize}
\item 
\hypertarget{class_zegar_af747dc3a9d58207618ec877990900b80}{void {\bfseries Start} ()}\label{class_zegar_af747dc3a9d58207618ec877990900b80}

\item 
\hypertarget{class_zegar_a8a88ddd1aa0768bfbe37217e32a01da0}{void {\bfseries Koniec} ()}\label{class_zegar_a8a88ddd1aa0768bfbe37217e32a01da0}

\item 
\hypertarget{class_zegar_a78715c1ac3a9b593c6fd5d1a17b79308}{double {\bfseries Wynik} ()}\label{class_zegar_a78715c1ac3a9b593c6fd5d1a17b79308}

\end{DoxyCompactItemize}


The documentation for this class was generated from the following files\-:\begin{DoxyCompactItemize}
\item 
C\-:/\-Users/\-Klijek/\-Desktop/\-L\-A\-B7/prj/\hyperlink{zegar_8hpp}{zegar.\-hpp}\item 
C\-:/\-Users/\-Klijek/\-Desktop/\-L\-A\-B7/prj/\hyperlink{zegar_8cpp}{zegar.\-cpp}\end{DoxyCompactItemize}

\chapter{Dokumentacja plików}
\hypertarget{main_8cpp}{\section{Dokumentacja pliku C\-:/\-Users/\-Klijek/\-Desktop/\-L\-A\-B4/prj/main.cpp}
\label{main_8cpp}\index{C\-:/\-Users/\-Klijek/\-Desktop/\-L\-A\-B4/prj/main.\-cpp@{C\-:/\-Users/\-Klijek/\-Desktop/\-L\-A\-B4/prj/main.\-cpp}}
}
{\ttfamily \#include $<$iostream$>$}\\*
{\ttfamily \#include \char`\"{}zegar.\-hpp\char`\"{}}\\*
{\ttfamily \#include \char`\"{}stos.\-hpp\char`\"{}}\\*
{\ttfamily \#include \char`\"{}kolejka.\-hpp\char`\"{}}\\*
{\ttfamily \#include \char`\"{}plik.\-hpp\char`\"{}}\\*
Wykres zależności załączania dla main.\-cpp\-:
\subsection*{Funkcje}
\begin{DoxyCompactItemize}
\item 
int \hyperlink{main_8cpp_ae66f6b31b5ad750f1fe042a706a4e3d4}{main} ()
\end{DoxyCompactItemize}


\subsection{Dokumentacja funkcji}
\hypertarget{main_8cpp_ae66f6b31b5ad750f1fe042a706a4e3d4}{\index{main.\-cpp@{main.\-cpp}!main@{main}}
\index{main@{main}!main.cpp@{main.\-cpp}}
\subsubsection[{main}]{\setlength{\rightskip}{0pt plus 5cm}int main (
\begin{DoxyParamCaption}
{}
\end{DoxyParamCaption}
)}}\label{main_8cpp_ae66f6b31b5ad750f1fe042a706a4e3d4}


Definicja w linii 16 pliku main.\-cpp.



Oto graf wywołań dla tej funkcji\-:



\hypertarget{operacje_8cpp}{\section{Dokumentacja pliku prj/operacje.cpp}
\label{operacje_8cpp}\index{prj/operacje.\-cpp@{prj/operacje.\-cpp}}
}
{\ttfamily \#include \char`\"{}operacje.\-hpp\char`\"{}}\\*
Wykres zależności załączania dla operacje.\-cpp\-:
\nopagebreak
\begin{figure}[H]
\begin{center}
\leavevmode
\includegraphics[width=196pt]{operacje_8cpp__incl}
\end{center}
\end{figure}
\subsection*{Funkcje}
\begin{DoxyCompactItemize}
\item 
void \hyperlink{operacje_8cpp_a0e1dcf6202da97bb49029fd97c1dc6a9}{Zamien\-Ele} (\hyperlink{class_tablica}{Tablica} \&T, int m1, int m2)
\item 
void \hyperlink{operacje_8cpp_a0933bc7e6ecc1138cde73ad221d3a6a6}{Odwroc\-Tab} (\hyperlink{class_tablica}{Tablica} \&T)
\item 
void \hyperlink{operacje_8cpp_a622767484b449600cf6446da085c561e}{Dodaj\-Ele} (\hyperlink{class_tablica}{Tablica} \&T, double ele)
\item 
void \hyperlink{operacje_8cpp_afcebfcc0c075278505a19b1c9afff30d}{Dodaj\-Ele} (\hyperlink{class_tablica}{Tablica} \&T1, \hyperlink{class_tablica}{Tablica} \&T2, double ele)
\end{DoxyCompactItemize}


\subsection{Dokumentacja funkcji}
\hypertarget{operacje_8cpp_a622767484b449600cf6446da085c561e}{\index{operacje.\-cpp@{operacje.\-cpp}!Dodaj\-Ele@{Dodaj\-Ele}}
\index{Dodaj\-Ele@{Dodaj\-Ele}!operacje.cpp@{operacje.\-cpp}}
\subsubsection[{Dodaj\-Ele}]{\setlength{\rightskip}{0pt plus 5cm}void Dodaj\-Ele (
\begin{DoxyParamCaption}
\item[{{\bf Tablica} \&}]{T, }
\item[{double}]{ele}
\end{DoxyParamCaption}
)}}\label{operacje_8cpp_a622767484b449600cf6446da085c561e}


Definicja w linii 14 pliku operacje.\-cpp.



Oto graf wywołań dla tej funkcji\-:
\nopagebreak
\begin{figure}[H]
\begin{center}
\leavevmode
\includegraphics[width=294pt]{operacje_8cpp_a622767484b449600cf6446da085c561e_cgraph}
\end{center}
\end{figure}




Oto graf wywoływań tej funkcji\-:
\nopagebreak
\begin{figure}[H]
\begin{center}
\leavevmode
\includegraphics[width=210pt]{operacje_8cpp_a622767484b449600cf6446da085c561e_icgraph}
\end{center}
\end{figure}


\hypertarget{operacje_8cpp_afcebfcc0c075278505a19b1c9afff30d}{\index{operacje.\-cpp@{operacje.\-cpp}!Dodaj\-Ele@{Dodaj\-Ele}}
\index{Dodaj\-Ele@{Dodaj\-Ele}!operacje.cpp@{operacje.\-cpp}}
\subsubsection[{Dodaj\-Ele}]{\setlength{\rightskip}{0pt plus 5cm}void Dodaj\-Ele (
\begin{DoxyParamCaption}
\item[{{\bf Tablica} \&}]{T1, }
\item[{{\bf Tablica} \&}]{T2, }
\item[{double}]{ele}
\end{DoxyParamCaption}
)}}\label{operacje_8cpp_afcebfcc0c075278505a19b1c9afff30d}


Definicja w linii 20 pliku operacje.\-cpp.



Oto graf wywołań dla tej funkcji\-:
\nopagebreak
\begin{figure}[H]
\begin{center}
\leavevmode
\includegraphics[width=294pt]{operacje_8cpp_afcebfcc0c075278505a19b1c9afff30d_cgraph}
\end{center}
\end{figure}


\hypertarget{operacje_8cpp_a0933bc7e6ecc1138cde73ad221d3a6a6}{\index{operacje.\-cpp@{operacje.\-cpp}!Odwroc\-Tab@{Odwroc\-Tab}}
\index{Odwroc\-Tab@{Odwroc\-Tab}!operacje.cpp@{operacje.\-cpp}}
\subsubsection[{Odwroc\-Tab}]{\setlength{\rightskip}{0pt plus 5cm}void Odwroc\-Tab (
\begin{DoxyParamCaption}
\item[{{\bf Tablica} \&}]{T}
\end{DoxyParamCaption}
)}}\label{operacje_8cpp_a0933bc7e6ecc1138cde73ad221d3a6a6}


Definicja w linii 9 pliku operacje.\-cpp.



Oto graf wywołań dla tej funkcji\-:
\nopagebreak
\begin{figure}[H]
\begin{center}
\leavevmode
\includegraphics[width=288pt]{operacje_8cpp_a0933bc7e6ecc1138cde73ad221d3a6a6_cgraph}
\end{center}
\end{figure}




Oto graf wywoływań tej funkcji\-:
\nopagebreak
\begin{figure}[H]
\begin{center}
\leavevmode
\includegraphics[width=222pt]{operacje_8cpp_a0933bc7e6ecc1138cde73ad221d3a6a6_icgraph}
\end{center}
\end{figure}


\hypertarget{operacje_8cpp_a0e1dcf6202da97bb49029fd97c1dc6a9}{\index{operacje.\-cpp@{operacje.\-cpp}!Zamien\-Ele@{Zamien\-Ele}}
\index{Zamien\-Ele@{Zamien\-Ele}!operacje.cpp@{operacje.\-cpp}}
\subsubsection[{Zamien\-Ele}]{\setlength{\rightskip}{0pt plus 5cm}void Zamien\-Ele (
\begin{DoxyParamCaption}
\item[{{\bf Tablica} \&}]{T, }
\item[{int}]{m1, }
\item[{int}]{m2}
\end{DoxyParamCaption}
)}}\label{operacje_8cpp_a0e1dcf6202da97bb49029fd97c1dc6a9}


Definicja w linii 4 pliku operacje.\-cpp.



Oto graf wywołań dla tej funkcji\-:
\nopagebreak
\begin{figure}[H]
\begin{center}
\leavevmode
\includegraphics[width=282pt]{operacje_8cpp_a0e1dcf6202da97bb49029fd97c1dc6a9_cgraph}
\end{center}
\end{figure}




Oto graf wywoływań tej funkcji\-:
\nopagebreak
\begin{figure}[H]
\begin{center}
\leavevmode
\includegraphics[width=218pt]{operacje_8cpp_a0e1dcf6202da97bb49029fd97c1dc6a9_icgraph}
\end{center}
\end{figure}



\hypertarget{operacje_8hpp}{\section{Dokumentacja pliku prj/operacje.hpp}
\label{operacje_8hpp}\index{prj/operacje.\-hpp@{prj/operacje.\-hpp}}
}


Definicja funkcji Zamien\-Ele.  


{\ttfamily \#include \char`\"{}tablica.\-hpp\char`\"{}}\\*
Wykres zależności załączania dla operacje.\-hpp\-:
\nopagebreak
\begin{figure}[H]
\begin{center}
\leavevmode
\includegraphics[width=196pt]{operacje_8hpp__incl}
\end{center}
\end{figure}
Ten wykres pokazuje, które pliki bezpośrednio lub pośrednio załączają ten plik\-:
\nopagebreak
\begin{figure}[H]
\begin{center}
\leavevmode
\includegraphics[width=254pt]{operacje_8hpp__dep__incl}
\end{center}
\end{figure}
\subsection*{Funkcje}
\begin{DoxyCompactItemize}
\item 
void \hyperlink{operacje_8hpp_a0e1dcf6202da97bb49029fd97c1dc6a9}{Zamien\-Ele} (\hyperlink{class_tablica}{Tablica} \&T, int m1, int m2)
\item 
void \hyperlink{operacje_8hpp_a0933bc7e6ecc1138cde73ad221d3a6a6}{Odwroc\-Tab} (\hyperlink{class_tablica}{Tablica} \&T)
\item 
void \hyperlink{operacje_8hpp_a622767484b449600cf6446da085c561e}{Dodaj\-Ele} (\hyperlink{class_tablica}{Tablica} \&T, double ele)
\item 
void \hyperlink{operacje_8hpp_afcebfcc0c075278505a19b1c9afff30d}{Dodaj\-Ele} (\hyperlink{class_tablica}{Tablica} \&T1, \hyperlink{class_tablica}{Tablica} \&T2, double ele)
\end{DoxyCompactItemize}


\subsection{Opis szczegółowy}
Definicja funkcji Zamien\-Ele. Definicja funkcji Dodaj\-Ele.

Definicja funkcji Odwroc\-Tab.

Funkcja, ktora zamienia elementy.

Funkcja, ktora odwraca tablice.

Funkcja, ktora dodaje element.

Funkcja, ktora dodaje elementy. 

Definicja w pliku \hyperlink{operacje_8hpp_source}{operacje.\-hpp}.



\subsection{Dokumentacja funkcji}
\hypertarget{operacje_8hpp_a622767484b449600cf6446da085c561e}{\index{operacje.\-hpp@{operacje.\-hpp}!Dodaj\-Ele@{Dodaj\-Ele}}
\index{Dodaj\-Ele@{Dodaj\-Ele}!operacje.hpp@{operacje.\-hpp}}
\subsubsection[{Dodaj\-Ele}]{\setlength{\rightskip}{0pt plus 5cm}void Dodaj\-Ele (
\begin{DoxyParamCaption}
\item[{{\bf Tablica} \&}]{T, }
\item[{double}]{ele}
\end{DoxyParamCaption}
)}}\label{operacje_8hpp_a622767484b449600cf6446da085c561e}


Definicja w linii 14 pliku operacje.\-cpp.



Oto graf wywołań dla tej funkcji\-:
\nopagebreak
\begin{figure}[H]
\begin{center}
\leavevmode
\includegraphics[width=294pt]{operacje_8hpp_a622767484b449600cf6446da085c561e_cgraph}
\end{center}
\end{figure}




Oto graf wywoływań tej funkcji\-:
\nopagebreak
\begin{figure}[H]
\begin{center}
\leavevmode
\includegraphics[width=210pt]{operacje_8hpp_a622767484b449600cf6446da085c561e_icgraph}
\end{center}
\end{figure}


\hypertarget{operacje_8hpp_afcebfcc0c075278505a19b1c9afff30d}{\index{operacje.\-hpp@{operacje.\-hpp}!Dodaj\-Ele@{Dodaj\-Ele}}
\index{Dodaj\-Ele@{Dodaj\-Ele}!operacje.hpp@{operacje.\-hpp}}
\subsubsection[{Dodaj\-Ele}]{\setlength{\rightskip}{0pt plus 5cm}void Dodaj\-Ele (
\begin{DoxyParamCaption}
\item[{{\bf Tablica} \&}]{T1, }
\item[{{\bf Tablica} \&}]{T2, }
\item[{double}]{ele}
\end{DoxyParamCaption}
)}}\label{operacje_8hpp_afcebfcc0c075278505a19b1c9afff30d}


Definicja w linii 20 pliku operacje.\-cpp.



Oto graf wywołań dla tej funkcji\-:
\nopagebreak
\begin{figure}[H]
\begin{center}
\leavevmode
\includegraphics[width=294pt]{operacje_8hpp_afcebfcc0c075278505a19b1c9afff30d_cgraph}
\end{center}
\end{figure}


\hypertarget{operacje_8hpp_a0933bc7e6ecc1138cde73ad221d3a6a6}{\index{operacje.\-hpp@{operacje.\-hpp}!Odwroc\-Tab@{Odwroc\-Tab}}
\index{Odwroc\-Tab@{Odwroc\-Tab}!operacje.hpp@{operacje.\-hpp}}
\subsubsection[{Odwroc\-Tab}]{\setlength{\rightskip}{0pt plus 5cm}void Odwroc\-Tab (
\begin{DoxyParamCaption}
\item[{{\bf Tablica} \&}]{T}
\end{DoxyParamCaption}
)}}\label{operacje_8hpp_a0933bc7e6ecc1138cde73ad221d3a6a6}


Definicja w linii 9 pliku operacje.\-cpp.



Oto graf wywołań dla tej funkcji\-:
\nopagebreak
\begin{figure}[H]
\begin{center}
\leavevmode
\includegraphics[width=288pt]{operacje_8hpp_a0933bc7e6ecc1138cde73ad221d3a6a6_cgraph}
\end{center}
\end{figure}




Oto graf wywoływań tej funkcji\-:
\nopagebreak
\begin{figure}[H]
\begin{center}
\leavevmode
\includegraphics[width=222pt]{operacje_8hpp_a0933bc7e6ecc1138cde73ad221d3a6a6_icgraph}
\end{center}
\end{figure}


\hypertarget{operacje_8hpp_a0e1dcf6202da97bb49029fd97c1dc6a9}{\index{operacje.\-hpp@{operacje.\-hpp}!Zamien\-Ele@{Zamien\-Ele}}
\index{Zamien\-Ele@{Zamien\-Ele}!operacje.hpp@{operacje.\-hpp}}
\subsubsection[{Zamien\-Ele}]{\setlength{\rightskip}{0pt plus 5cm}void Zamien\-Ele (
\begin{DoxyParamCaption}
\item[{{\bf Tablica} \&}]{T, }
\item[{int}]{m1, }
\item[{int}]{m2}
\end{DoxyParamCaption}
)}}\label{operacje_8hpp_a0e1dcf6202da97bb49029fd97c1dc6a9}


Definicja w linii 4 pliku operacje.\-cpp.



Oto graf wywołań dla tej funkcji\-:
\nopagebreak
\begin{figure}[H]
\begin{center}
\leavevmode
\includegraphics[width=282pt]{operacje_8hpp_a0e1dcf6202da97bb49029fd97c1dc6a9_cgraph}
\end{center}
\end{figure}




Oto graf wywoływań tej funkcji\-:
\nopagebreak
\begin{figure}[H]
\begin{center}
\leavevmode
\includegraphics[width=218pt]{operacje_8hpp_a0e1dcf6202da97bb49029fd97c1dc6a9_icgraph}
\end{center}
\end{figure}



\hypertarget{tablica_8cpp}{\section{Dokumentacja pliku prj/tablica.cpp}
\label{tablica_8cpp}\index{prj/tablica.\-cpp@{prj/tablica.\-cpp}}
}


Definicja metody Dodaj\-Na\-Koniec.  


{\ttfamily \#include \char`\"{}tablica.\-hpp\char`\"{}}\\*
{\ttfamily \#include $<$fstream$>$}\\*
Wykres zależności załączania dla tablica.\-cpp\-:
\nopagebreak
\begin{figure}[H]
\begin{center}
\leavevmode
\includegraphics[width=235pt]{tablica_8cpp__incl}
\end{center}
\end{figure}


\subsection{Opis szczegółowy}
Definicja metody Dodaj\-Na\-Koniec. Definicja przeladowania operatora == .

Definicja przeladowania operatora = .

Definicja przeladowania operatora + .

Definicja metody Usun\-Ele.

Definicja metody Wyswietl\-Tab.

Definicja metody Pobierz\-Wskazany\-Ele.

Definicja metody Dodaj\-Na\-Miejsce.

Definicja metody Odwroc\-Tab.

Definicja metody zamien elementy.

Definicja metody Pobierz\-Rozmiar.

Definicja metody Pobierz\-Ostatni\-Ele.

Definicja metody Pobierz\-Pierwszy\-Ele.

Definicja metody Dodaj\-Na\-Poczatek.

Metoda, ktora dodaje na koniec element.

Metoda, ktora dodaje na poczatek element.

Metoda, ktora pobiera pierwszy element.

Metoda, ktora pobiera ostatni element.

Metoda, ktora pobiera pobiera rozmiar tablicy.

Metoda, ktora zamienia dwa wybrane elementy.

Metoda, ktora odwraca tablice.

Metoda, ktora dodaje element na konretne miejsce.

Metoda, ktora pobiera wskazany element.

Metoda, ktora wyswietla tabilce.

Metoda, ktora dusuwa elementy.

W celu lacznia dwoch tablic w jedna.

W celu przypisania dwoch tablic.

W celu lacznia porownania tablic. 

Definicja w pliku \hyperlink{tablica_8cpp_source}{tablica.\-cpp}.


\hypertarget{tablica_8hpp}{\section{Dokumentacja pliku prj/tablica.hpp}
\label{tablica_8hpp}\index{prj/tablica.\-hpp@{prj/tablica.\-hpp}}
}


Definicja klasy \hyperlink{class_tablica}{Tablica}.  


{\ttfamily \#include $<$vector$>$}\\*
{\ttfamily \#include $<$iostream$>$}\\*
Wykres zależności załączania dla tablica.\-hpp\-:
\nopagebreak
\begin{figure}[H]
\begin{center}
\leavevmode
\includegraphics[width=196pt]{tablica_8hpp__incl}
\end{center}
\end{figure}
Ten wykres pokazuje, które pliki bezpośrednio lub pośrednio załączają ten plik\-:
\nopagebreak
\begin{figure}[H]
\begin{center}
\leavevmode
\includegraphics[width=323pt]{tablica_8hpp__dep__incl}
\end{center}
\end{figure}
\subsection*{Komponenty}
\begin{DoxyCompactItemize}
\item 
class \hyperlink{class_tablica}{Tablica}
\end{DoxyCompactItemize}


\subsection{Opis szczegółowy}
Definicja klasy \hyperlink{class_tablica}{Tablica}. Klasa odpowiadaj�ca za operacje wykonujace sie na tablicy takie jak\-: zamienienie elementow, odwrocenie tablicy, dodanie elementu, dodanie elementow, operator + operator, operato == 
\begin{DoxyParams}[1]{Parametry}
\mbox{\tt in}  & {\em tab} & -\/ kontener dynamicznej tablicy jednowymiarowej o typie double \\
\hline
\end{DoxyParams}


Definicja w pliku \hyperlink{tablica_8hpp_source}{tablica.\-hpp}.


\hypertarget{zegar_8cpp}{\section{C\-:/\-Users/\-Klijek/\-Desktop/\-L\-A\-B7/prj/zegar.cpp File Reference}
\label{zegar_8cpp}\index{C\-:/\-Users/\-Klijek/\-Desktop/\-L\-A\-B7/prj/zegar.\-cpp@{C\-:/\-Users/\-Klijek/\-Desktop/\-L\-A\-B7/prj/zegar.\-cpp}}
}


Definicja metody Start.  


{\ttfamily \#include \char`\"{}zegar.\-hpp\char`\"{}}\\*
Include dependency graph for zegar.\-cpp\-:


\subsection{Detailed Description}
Definicja metody Start. Definicja metody Wynik.

Definicja metody Koniec.

Metoda, ktora powoduje start zegara.

Metoda, ktora powoduje stop zegara i oblicza czas.

Metoda, ktora wyswietla wynik. 
\hypertarget{zegar_8hpp}{\section{C\-:/\-Users/\-Klijek/\-Desktop/\-L\-A\-B7/prj/zegar.hpp File Reference}
\label{zegar_8hpp}\index{C\-:/\-Users/\-Klijek/\-Desktop/\-L\-A\-B7/prj/zegar.\-hpp@{C\-:/\-Users/\-Klijek/\-Desktop/\-L\-A\-B7/prj/zegar.\-hpp}}
}


Definicja klasy Tablica.  


{\ttfamily \#include $<$ctime$>$}\\*
{\ttfamily \#include $<$iostream$>$}\\*
Include dependency graph for zegar.\-hpp\-:
This graph shows which files directly or indirectly include this file\-:
\subsection*{Classes}
\begin{DoxyCompactItemize}
\item 
class \hyperlink{class_zegar}{Zegar}
\end{DoxyCompactItemize}


\subsection{Detailed Description}
Definicja klasy Tablica. Klasa odpowiadaj�ca za operacje wykonujace sie na tablicy takie jak\-: zamienienie elementow, odwrocenie tablicy, dodanie elementu, dodanie elementow, operator + operator, operato == 
\begin{DoxyParams}[1]{Parameters}
\mbox{\tt in}  & {\em clock\-\_\-t} & start, koniec -\/ zmienne przechowuja aktualny czas systemu \\
\hline
\mbox{\tt in}  & {\em czas} & -\/ przechowuje roznice czasow koniec i start \\
\hline
\end{DoxyParams}

\printindex
\end{document}
