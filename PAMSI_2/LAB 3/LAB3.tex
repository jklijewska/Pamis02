\documentclass[11pt]{article}
 
\usepackage[T1]{fontenc}
\usepackage[polish]{babel}
\usepackage[utf8]{inputenc}
\usepackage{lmodern}
\usepackage{graphicx}
\selectlanguage{polish}

\begin{document}
\begin{titlepage}
\title{Kolejki i stosy - Laboratorium nr 3 z PAMSI}
\author{Justyna Klijewska}
\date{16 03 2014}
\maketitle
\end{titlepage}

\underline{Zadanie do wykonania}

Zaimplementować stos i kolejkę przy pomocy tablic i przy pomocy list. 
\\ \\ \\
Program był pisany w środowisku Windows. Jest to wersja na ocenę 5. Wykresy zostały wykonane w programie MatLab i wklejone jako obrazy. 
\\ \\ \\ \\ 
\textit{Stos: tablicowa }
\\ \\


\begin{tabular}{|r|l|}
  \hline 
  liczba elementow pliku & czas wykonania \\
  \hline
  10 & 0.0029 \\
  \hline
  100 & 0.23 \\
  \hline
  1000 & 0.77 \\
  \hline
  10000 & 9.47 \\
  \hline
\end{tabular} 
\\ Tabela 1. Zależności między liczbą elementów w pliku a czasem wykonywania programu.
\\
\begin{figure}[ht!]
\centering
\includegraphics[width=90mm]{TS.jpg}
\caption{Wykres zależności liczby elementów w pliku od czasu wykonywania programu}
\label{overflow}
\end{figure}


\textit{Stos: listowa }
\\ \\ \\
\begin{tabular}{|r|l|}
  \hline 
  liczba elementow pliku & czas wykonania \\
  \hline
  10 & 0.002 \\
  \hline
  100 & 0.19 \\
  \hline
  1000 & 0.96 \\
  \hline
  10000 & 8.73 \\
  \hline
\end{tabular} 
\\ Tabela 1. Zależności między liczbą elementów w pliku a czasem wykonywania programu.
\\
\begin{figure}[ht!]
\centering
\includegraphics[width=90mm]{LS.jpg}
\caption{Wykres zależności liczby elementów w pliku od czasu wykonywania programu}
\label{overflow}
\end{figure}

\textit{Kolejka: tablicowa }
\\ \\ \\
\begin{tabular}{|r|l|}
  \hline 
  liczba elementow pliku & czas wykonania \\
  \hline
  10 & 0.049 \\
  \hline
  100 & 0.23 \\
  \hline
  1000 & 0.97 \\
  \hline
  10000 & 9.05 \\
  \hline
\end{tabular} 
\\ Tabela 1. Zależności między liczbą elementów w pliku a czasem wykonywania programu.
\\
\begin{figure}[ht!]
\centering
\includegraphics[width=90mm]{TK.jpg}
\caption{Wykres zależności liczby elementów w pliku od czasu wykonywania programu}
\label{overflow}
\end{figure}

\textit{Kolejka: listowa }
\\ \\ \\
\begin{tabular}{|r|l|}
  \hline 
  liczba elementow pliku & czas wykonania \\
  \hline
  10 & 0.001 \\
  \hline
  100 & 0.13 \\
  \hline
  1000 & 0.92 \\
  \hline
  10000 & 8.75 \\
  \hline
\end{tabular} 
\\ Tabela 1. Zależności między liczbą elementów w pliku a czasem wykonywania programu.
\\
\begin{figure}[ht!]
\centering
\includegraphics[width=90mm]{TK.jpg}
\caption{Wykres zależności liczby elementów w pliku od czasu wykonywania programu}
\label{overflow}
\end{figure}
\\ \\ \\ \\ \\ \\ \\ \\ \\ \\ \\ \\ \\
\underline {WNIOSKI:} \\
Wykresy zależności bardzo przypominają charakterystyki liniowe. Dużo szybsze jest użycie kolejki. Czas wczytania stosu listowego jest szybszy niż wczytania stosu tablicowego. Jeśli chodzi o kolejkę to szybciej wykonuje się tablicowa.
\end{document}
