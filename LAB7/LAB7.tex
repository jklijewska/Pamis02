\documentclass[11pt]{article}
 
\usepackage[T1]{fontenc}
\usepackage[polish]{babel}
\usepackage[utf8]{inputenc}
\usepackage{lmodern}
\usepackage{graphicx}
\selectlanguage{polish}

\begin{document}
\begin{titlepage}
\title{Tablica asocjacyjna, haszująca oraz drzewa binarne BST - Laboratorium nr 7 z PAMSI}
\author{Justyna Klijewska}
\date{22 04 2014}
\maketitle
\end{titlepage}
\section{Wstęp}
Celem ćwiczenia było przetestowanie implementacji (zmierzenie poszczególnych czasów) za pomocą: \\
- tablicy asocjacyjnej (wykonanej w tym przypadku na liście) \\ -tablicy haszującej \\ -drzewa binarnego BST.
Program został przetestowany w środowisku Windows.

\section{Teoria}
\underline {Drzewo binarne} \\ \\
W teorii grafów jest to drzewo, w którym stopień każdego wierzchołka wynosi maksymalnie 2. W najgorszym przypadku jego złożoność jest liniowa - dzieje się tak, kiedy drzewo składa się tylko z jednej gałęzi. W przypadku optymistycznym tak konstruujemy drzewo, aby zawsze po lewej stronie znajdowały się elementy nie większe, a po prawej większe. Następnie wyszukując dany element odrzucamy zawsze po połowie naszych elementów, co wpływa na szybki czas.
\\ \\
\underline {Tablica haszująca} \\ \\
Ponieważ zmieszczenie wszystkich kluczy i elementów dużej tablicy w pamięci byłoby niemożliwe, stosujemy tablicę haszującą. Dzięki temu możemy uzyskać szybki dostęp do przechowywanych informacji. W najprostszym przypadku wartość funkcji mieszającej, obliczona dla danego klucza, wyznacza dokładnie indeks szukanej informacji w tablicy. Jeżeli miejsce wskazywane przez obliczony indeks jest puste, to poszukiwanej informacji nie ma w tablicy. W ten sposób wyszukiwanie elementu ma złożoność czasową O(1). Jednak w sytuacji tej pojawia się problem \textbf {kolizji}, to znaczy przypisania przez funkcję mieszającą tej samej wartości dwóm różnym kluczom. Pesymistyczny przypadek to taki, kiedy mamy liniową złożoność, a optymistyczny kiedy jest ona zależna tylko od czasu wykonywaniu funkcji haszującej. 
\\ \\ \\
\section{Tabele pomiarowe, wykresy}
\underline{Tablica asocjacyjna} \\ \\
\begin{tabular}{|c|c|}
  \hline
  Ilość danych & Czas [s] \\
  \hline \hline
  10 & 0 \\ \hline
  100 & 0.001\\ \hline
  1000 & 0.002 \\ \hline
  10000 & 0.031 \\ \hline
  100000 & 0.400 \\ \hline
  1000000 & 3.921 \\ \hline
  
  \hline
  \end{tabular} 
\\ \\ \\
\underline{Tablica haszująca} \\ \\
\begin{tabular}{|c|c|} 
  \hline
  Ilość danych & Czas [s] \\
  \hline \hline
  10 & 0 \\ \hline
  100 & 0.001\\  \hline
  1000 & 0.002\\  \hline
  10000 & 0.023\\  \hline
  100000 & 0.291 \\ \hline
  1000000 & 3.117 \\ \hline
  \hline
  \end{tabular} 
 \\ \\ \\
\underline{Drzewo binarne BST} \\ \\
\begin{tabular}{|c|c|}
  \hline
  Ilość danych & Czas [s] \\
  \hline \hline
  10 & 0 \\ \hline
  100 & 0.001\\  \hline
  1000 & 0.002\\  \hline
  10000 & 0.029\\  \hline
  100000 & 0.362 \\ \hline
  1000000 & 3.887 \\ \hline
  \hline
  \end{tabular} 
\\ \\ \\  \\ \\ \\ \\ \\ \\
 \underline{Wykres dla powyższych algorytmów}
 \\
\begin{figure}[ht!]
\centering
\includegraphics[width=90mm]{porow.jpg}
\caption{Wykres zależności liczby elementów od czasu implementacji dla powyższych algorytmów.}
\label{overflow}
\end{figure}

\section{Wnioski}
\begin{itemize}
\item Przedstawione powyżej algorytmy cechują się dużą szybkością (nawet dla milionowego rozmiaru danych wykonanie algorytmu nie trwa długo). 
\item Z powodu szybkiego wyszukiwania powyższe algorytmy wykorzystuje się przy tworzeniu słowników.
\item Przy dużej ilości danych, można zauważyć, że tablica haszująca jest efektywniejsza niż drzewo binarne. 
\item Powyższe algorytmy mają złożoność obliczeniową: \\ \\
\begin{tabular}{|c|c|c|c|}
  \hline
  Algorytm & Złożoność &Złożoność& Złożoność \\
  &Pesymistyczna & Typowa &Optymistyczna \\ \hline
  \hline 
  T. asocjacyjna&&O(log2n)& \\ \hline
  Drzewo binarne BST&O(n)&&O(log2n)\\ \hline
  T.haszująca&O(n)&&O(1) \\ \hline
  \hline
  \end{tabular} 
\end{itemize}

\end{document}